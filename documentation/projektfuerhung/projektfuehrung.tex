\documentclass[10pt,paper=a4,final]{scrartcl}
\usepackage[utf8]{inputenc}
\usepackage{tabularx}		%used for the tables
\usepackage{geometry}		%allows us to specify the 'seitenrand'
\usepackage[table]{xcolor}	%allows us to make colored fields in the tables
\usepackage{graphicx}		%package used to include graphics
\usepackage{hyperref}   	%used to make klickable links
\usepackage{pdflscape}	%allows landscape mode
\usepackage{supertabular}
\usepackage{listings}
\lstset{language=python,
  basicstyle=\small, 
  keywordstyle=\color{blue!80!black!100}, 
  identifierstyle=, 
  commentstyle=\color{green!50!black!100}, 
  stringstyle=\ttfamily\color{red!80!black!100}, 
  breaklines=true, 
  numbers=left, 
  numberstyle=\small, 
  frame=single, 
  backgroundcolor=\color{blue!3} 
} 

\setcounter{tocdepth}{4}        %include paragraph in tableofcontents
\setcounter{secnumdepth}{5}     %also number the paragraphs

%\hypersetup{linktocpage}	%make the tableofcontent klickable
\hypersetup{
  colorlinks,
  citecolor=black,
  filecolor=black,
  linkcolor=black,
  urlcolor=black
}

\setcounter{tocdepth}{4}	%include paragraph in tableofcontents
\setcounter{secnumdepth}{5}	%also number the paragraphs

%These two lines will allow us to specify our own headers/footers
\usepackage{fancyhdr}
\pagestyle{fancy}
 \setlength{\parskip}{0pt}
 \setlength{\baselineskip}{0pt}

%The next three lines set the default font to Arial
%use 'getnonfreefonts arial-urw' to install uarial on Linux systems
\usepackage[T1]{fontenc}
\usepackage[scaled]{uarial}
\renewcommand*\familydefault{\sfdefault}

\geometry{a4paper, top=20mm, right=20mm, bottom=20mm, left=20mm}
\title{Projektfuehrungsbericht}
\author{Niklaus Hofer, Lukas Kn\"opfel, Kaleb Tschabold}
\date{\today}

%defining header and footer
\fancyhf{}	%delete default values
\setlength{\headwidth}{\textwidth}	%header and footer width equal the text width
\fancyhead[LE,LO]{\includegraphics[scale=0.6]{header.png}}
\fancyhead[RE,RO]{ProjectExplorer}
\fancyfoot[CE,CO]{Speicherdatum: \today{}}
\fancyfoot[RE,RO]{\thepage}

\begin{document}
\maketitle
\newpage
\begin{tabularx}{\textwidth}{ r X }	%X fields are stretched over the whole space
  \textcolor{white}{{\bf Status}}\cellcolor{blue!80!} & In Arbeit/In Prüfung/ {\bf Abgeschlossen}\cellcolor{blue!20!} \\
\textcolor{white}{{\bf Projektname}}\cellcolor{blue!80!} & Projektexplorer\cellcolor{blue!20!} \\
\textcolor{white}{{\bf Projektleiter}}\cellcolor{blue!80!} & Lukas Kn\"opfel\cellcolor{blue!20!} \\
\textcolor{white}{{\bf Auftraggeber}}\cellcolor{blue!80!} & M. Frieden, GIBB\cellcolor{blue!20!} \\
\textcolor{white}{{\bf Autoren}}\cellcolor{blue!80!} & Kaleb Tschabold, Lukas Kn\"opfel, Niklaus Hofer\cellcolor{blue!20!} \\
\textcolor{white}{{\bf Verteiler}}\cellcolor{blue!80!} & Lukas Knöpfel, Kaleb Tschabolt, Niklaus Hofer\cellcolor{blue!20!}
\end{tabularx}
\newline
\newline
\newline
{\bf Änderungskontrolle, Prüfung, Genehmigung}
\newline

\begin{tabularx}{\textwidth}{l l X X}
\textcolor{white}{Version}\cellcolor{blue!80!} & \textcolor{white}{Datum}\cellcolor{blue!80!} & \textcolor{white}{Beschreibung, Bemerkung}\cellcolor{blue!80!} & \textcolor{white}{Name oder Rolle}\cellcolor{blue!80!} \\
\cellcolor{blue!20!} 0.1& \cellcolor{blue!20!} 08.02.2011 & Gesammelten Text einfügen \cellcolor{blue!20!} & Kaleb Tschabold \cellcolor{blue!20!} \\
\cellcolor{blue!20!} 0.9& \cellcolor{blue!20!} 08.02.2011 & Abgabebereit \cellcolor{blue!20!} & Kaleb Tschabold \cellcolor{blue!20!} \\
\cellcolor{blue!20!} 0.99& \cellcolor{blue!20!} \today{} & Transfer nach \LaTeX \cellcolor{blue!20!} & Niklaus Hofer \cellcolor{blue!20!} \\
\end{tabularx}
\newline
\newline
\newline
{\bf Definitionen und Abkürzungen}
\newline

\begin{tabularx}{\textwidth}{l X}
\textcolor{white}{Begriff/ Abkürzung}\cellcolor{blue!80!} & \textcolor{white}{Bedeutung}\cellcolor{blue!80!} \\
CLI \cellcolor{blue!20!} & Command Line Interface\cellcolor{blue!20!} \\
GUI \cellcolor{blue!20!} & Graphical user interface \cellcolor{blue!20!} \\
DB \cellcolor{blue!20!} & Database\cellcolor{blue!20!} \\
\end{tabularx}
\newline
\newline
\newline
\bibliographystyle{plain}
\bibliography{projektfuehrung}{}
\flushleft
\newpage
\tableofcontents
\newpage
\section{Zweck des Dokuemnts}
Zusammenfassung von Planung und Ergebnissen zu den fünf Projektführungsthemen.
\begin{itemize}
  \item Projektmanagement
  \item Risikomanagement
  \item Qualitätsmanagement
  \item Konfigurationsmanagement
  \item Projektmarketing
\end{itemize}
\section{Projektplan}
\newpage
\begin{landscape}
\begin{tabularx}{\textwidth}{ |p{4.0cm}|l|l|l|l|l|l|l|l|l|l|l|l|l|l|l|l|l|l|l|l|l|l|l| }
\cline{1-24}	%need to use cline, cuz hline did not stretch over the whole width...
\bf Aktivit\"at & PL Soll & PL Ist & KW 05 & 06 & 07 & 08 & 09 & 10 & 11 & 12 & 13 & 14 & 15 & 16 & 17 & 18 & 19 & 20 & 21 & 22 & 23 & 24 & 25 \\
\cline{1-24}
Initialisierung & 12 & 12 & & \cellcolor[gray]{0.7} & & & & & & & & & & & & & & & & & & & \\
\cline{1-24}
Voranalyse & 36 & 36 & & & \cellcolor[gray]{0.7} & \cellcolor[gray]{0.7} & \cellcolor[gray]{0.7} & & & & & & & & & & & & & & & & \\
\cline{1-24}
Konzept & 36 & 36 & & & & & & \cellcolor[gray]{0.7} & \cellcolor[gray]{0.7} & \cellcolor[gray]{0.7} & & & & & & & & & & & & & \\
\cline{1-24}
Realisierung & 48 & 192 \cellcolor{red!100!} & & & & & & & & & \cellcolor[gray]{0.7} & \multicolumn{3}{|c|}{Ferien \cellcolor{green!70!}} & \cellcolor[gray]{0.7} & \cellcolor[gray]{0.7} & \cellcolor[gray]{0.7} & & & & & & \\
\cline{1-24}
Einf\"uhrung & 24 & 24 & & & & & & & & & & & & & & & & \cellcolor[gray]{0.7} & \cellcolor[gray]{0.7} & & & & \\
\cline{1-24}
Reserve & 24 & - & & & & & & & & & & & & & & & & & & \cellcolor[gray]{0.7} & \cellcolor[gray]{0.7} & & \\
\cline{1-24}
Abschlussphase & 12 & 24 \cellcolor{red!100!} & & & & & & & & & & & & & & & & & & & & \cellcolor[gray]{0.7} & \cellcolor[gray]{0.7}\\
\cline{1-24}
\multicolumn{24}{|c|}{\bf Projektleiter} \\
\cline{1-24}
Niklaus Hofer & & & \cellcolor[gray]{0.7} & \cellcolor[gray]{0.7} & \cellcolor[gray]{0.7} & \cellcolor[gray]{0.7} & \cellcolor[gray]{0.7} & & & & & & & & & & & & & & & & \\
\cline{1-24}
Kaleb Tschabold & & & & & & & & \cellcolor[gray]{0.7} & \cellcolor[gray]{0.7} & \cellcolor[gray]{0.7} & \cellcolor[gray]{0.7} & \cellcolor[gray]{0.7} & \cellcolor[gray]{0.7} & \cellcolor[gray]{0.7} & \cellcolor[gray]{0.7} & & & & & & & & \\
\cline{1-24}
Lukas Kn\"opfel & & & & & & & & & & & & & & & & \cellcolor[gray]{0.7} & \cellcolor[gray]{0.7} & \cellcolor[gray]{0.7} & \cellcolor[gray]{0.7} & \cellcolor[gray]{0.7} & \cellcolor[gray]{0.7} & \cellcolor[gray]{0.7} & \cellcolor[gray]{0.7}\\
\cline{1-24}
\end{tabularx}
\end{landscape}
\newpage
\section{Projektbericht}
\subsection{KW 7}
\begin{description}
  \item {\bf Stand der Arbeit: } \\
    \begin{itemize}
      \item Nachforschungen zu verschidenen Technologien.
      \item Die Arbeiten am Voranalysebericht schreiten gut voran.
      \item Wir haben bereits ein recht gutes Bild der aktuellen Lage geschaffen und wissen in etwa, welche Produkte vergleichbares liefern.
    \end{itemize}
  \item {\bf Probleme und Fragen: } \\
    \begin{itemize}
      \item Zurzeit gibt es von unserer Seite her keine Fragen.
    \end{itemize}
  \item {\bf nächste Schritte: } \\
    \begin{itemize}
      \item Nächsten Dienstag werden wir die Anforderungen erarbeiten.
      \item Sobald wir uns zu den verschiedenen Technologien, die für das Projekt in Frage kommen informiert haben (voraussichtlich nächste Woche, hängt von den Vortschritten bei den Specs ab), werden wir die verschiedenen Möglichkeiten analysieren und vergleichen.
    \end{itemize}
\end{description}
\subsection{KW 8}
\begin{description}
  \item {\bf Stand der Arbeit: } \\
    \begin{itemize}
      \item Wir sind zur Zeit mit der Evaluierung der verschiedenen Varianten beschäftigt. Dabei sind die mit Ihnen bereits besprochenen Probleme aufgetaucht, die wir nun als nächsten Schritt lösen werden.
      \item Die Arbeit am Voranalysebericht schreitet ansonsten gut voran und ist beinahe abgeschlossen.
      \item Parallel zu dem neuen Dokument arbeiten wir zur Zeit auch an der Korrektur der alten Unterlagen.
    \end{itemize}
  \item {\bf Probleme und Fragen: } \\
    \begin{itemize}
      \item Zurzeit haben wir keine weiteren Fragen.
    \end{itemize}
  \item {\bf nächste Schritte: } \\
    \begin{itemize}
      \item Der nächste konkrete Schritt ist das Anpassen der Versionsanalyse.
      \item Ist die Voranalyse abgeschlossen wird das Dokument aufgeräumt (Rechtschreibung, Punktierung,\ldots) und dann mit LaTeX in die Reinversion geschrieben.
    \end{itemize}
\end{description}
\subsection{KW 9}
\begin{description}
  \item {\bf Stand der Arbeit: } \\
    \begin{itemize}
      \item Voranalysebericht fertig gestellt.
      \item Parallel zu den neuen Dokumenten arbeiten wir zur Zeit auch an der Korrektur der alten Unterlagen.
    \end{itemize}
  \item {\bf Probleme und Fragen: } \\
    \begin{itemize}
      \item Zurzeit haben wir keine weiteren Fragen.
    \end{itemize}
  \item {\bf nächste Schritte: } \\
    \begin{itemize}
      \item Erstellen des Konzeptberichtes.
    \end{itemize}
\end{description}
\subsection{KW 10}
\begin{description}
  \item {\bf Stand der Arbeit: } \\
    \begin{itemize}
      \item Ersellung des Konzeptberichts.
      \item Arbeit am Klassendiagram.
    \end{itemize}
  \item {\bf Probleme und Fragen: } \\
    \begin{itemize}
      \item Zurzeit haben wir keine weiteren Fragen.
    \end{itemize}
  \item {\bf nächste Schritte: } \\
    \begin{itemize}
      \item Die Arbeit am Konzepbericht wird noch eine Menge Arbeit beanspruchen. Besonders muessen viele Funktionen genau geplant und durchdacht werden.
    \end{itemize}
\end{description}
\subsection{KW 11}
\begin{description}
  \item {\bf Stand der Arbeit: } \\
    \begin{itemize}
      \item Ich(Kaleb) habe nun die Aufgabe des Projektleiters übernommen.
      \item Wir haben heute zum grossen Teil einige Prototypen erstellt.
      \item Auch haben wir unser Klassen-Diagramm überarbeitet.
    \end{itemize}
  \item {\bf Probleme und Fragen: } \\
    \begin{itemize}
      \item Zurzeit haben wir keine Fragen.
    \end{itemize}
  \item {\bf nächste Schritte: } \\
    \begin{itemize}
      \item Da wir mit unserem Konzept noch nicht weit sind, müssen wir jetzt vor allem zu Hause das Konzept erarbeiten.
      \item Besonders das Klassendiagramm wollen wir optimieren.
      \item Auch die kleinen Korrekturen von den vorherigen Dokumenten müssen wir noch nachführen(Projektantrag und Voranalyse).
    \end{itemize}
\end{description}
\subsection{KW 12}
\begin{description}
  \item {\bf Stand der Arbeit: } \\
    \begin{itemize}
      \item Wir haben gestern das Konzept abgeschlossen. Wir haben gemerkt, dass wir den Zeitplan nicht gut eingeteilt haben und darum sind wir gestern ein bisschen in Zeitdruck gekommen.
      \item Wir beginnen jetzt mit der Realisierung.
    \end{itemize}
  \item {\bf Probleme und Fragen: } \\
    \begin{itemize}
      \item Zurzeit haben wir keine Fragen.
    \end{itemize}
  \item {\bf nächste Schritte: } \\
    \begin{itemize}
      \item Wir werden nun nächstes Mal, die Aufteilung machen, in der definiert ist, wer welche Programmteile programmiert.
      \item In den Frühlingsferien haben wir dann Zeit um voll am Projekt zu arbeiten.
    \end{itemize}
\end{description}
\subsection{KW 13}
\begin{description}
  \item {\bf Stand der Arbeit: } \\
    \begin{itemize}
      \item Wir sind gerade dran richtig in die Realisierungsphase zu starten.
      \item Wir sind gerade dran alle Dateien und etc. zu erstellen, damit wir, dann die Funktionalitäten erstellen können.
    \end{itemize}
  \item {\bf Probleme und Fragen: } \\
    \begin{itemize}
      \item Zurzeit haben wir keine Fragen.
    \end{itemize}
  \item {\bf nächste Schritte: } \\
    \begin{itemize}
      \item Ich werde einen Zeitplan für die Unterrichtsfreiezeit machen so, dass wir gegen Ende dieser Phase nicht in Zeitdruck kommen.
      \item Wir müssen diese Zeit auch nutzen um unsere Dokumente(Voranlyse,\ldots) zu korrigieren.
      \item Auch die aktuellen Dokumente(Realisierungsbericht und Projektführungsbericht) ergänzen.
    \end{itemize}
\end{description}
\subsection{KW 17}
\begin{description}
  \item {\bf Stand der Arbeit: } \\
    \begin{itemize}
      \item Wir sind gerade voll dran am System entwickeln.
      \item Ich habe gerade heute Abend die erste Version zum Setzten und Suchen von Tags programmiert.
    \end{itemize}
  \item {\bf Probleme und Fragen: } \\
    \begin{itemize}
      \item Zurzeit haben wir keine Fragen.
    \end{itemize}
  \item {\bf nächste Schritte: } \\
    \begin{itemize}
      \item Wir müssen besprechen wer, was bis zum nächsten Meilenstein macht.
      \item Auch den Realisierungsbericht müssen wir anfangen zu erstellen.
    \end{itemize}
\end{description}
\subsection{KW 18}
\begin{description}
  \item {\bf Stand der Arbeit: } \\
    \begin{itemize}
      \item Ich (Lukas) habe nun die Aufgabe des Projektleiters übernommen.
      \item Wir haben eine experimentelle Version unseres Programmes die schon einige Grundfunktionen integriert hat.
    \end{itemize}
  \item {\bf Probleme und Fragen: } \\
    \begin{itemize}
      \item Zurzeit haben wir keine Fragen.
    \end{itemize}
  \item {\bf nächste Schritte: } \\
    \begin{itemize}
      \item Wir werden hauptsächlich am Programm arbeiten.
      \item Aus Zeitmangel werden wir uns auf die Linux Version unseres Programmes konzentrieren. Die Windows und Mac Versionen werden je nachdem auf wieviele Probleme wir stossen weniger Funktionalität besitzen als die Linux Version.
    \end{itemize}
\end{description}
\subsection{KW 20}
\begin{description}
  \item {\bf Stand der Arbeit: } \\
    \begin{itemize}
      \item Wir haben eine stabile Version unseres Programmes auf Linux.
      \item Wir haben uns einen Überblick über das Projekt verschafft. Wir wissen nun welche Funktionen funktionieren und welche eine Überarbeitung benötigen.
    \end{itemize}
  \item {\bf Probleme und Fragen: } \\
    \begin{itemize}
      \item Zurzeit haben wir keine Fragen.
    \end{itemize}
  \item {\bf nächste Schritte: } \\
    \begin{itemize}
      \item Wir werden alle Funktionen in das GUI implementieren, damit wir sie an der Präsentation zeigen können.
      \item Ausserdem werden wir das Programm noch vollständig auf Windows portieren und eine Versionierung einbauen.
      \item Wir bereiten uns auch auf die Präsentation vor, da diese nun vorverschoben wurde.
    \end{itemize}
\end{description}
\subsection{KW 21}
\begin{description}
  \item {\bf Stand der Arbeit: } \\
    \begin{itemize}
      \item Wir arbeiten an der Präsentation um diese wie vereinbart nächsten Dienstag Vorzutragen.
    \end{itemize}
  \item {\bf Probleme und Fragen: } \\
    \begin{itemize}
      \item Zurzeit haben wir keine Fragen.
    \end{itemize}
  \item {\bf nächste Schritte: } \\
    \begin{itemize}
      \item Wir werden die Präsentation fertigstellen.
      \item Wir werden uns absprechen um im vorgegebenen Zeitrahmen zu bleiben.
      \item Wir werden eine dedizierte Maschine aufsetzen um unser Tool zu demonstrieren.
    \end{itemize}
\end{description}
\section{Risikokatalog}
\begin{tabularx}{\textwidth}{|l|X|X|X|p{1cm}|}
  \hline
  \bf Nr. \cellcolor{blue!20!}& \bf Risiko f\"ur das Projekt \cellcolor{blue!20!}& \bf m\"ogliche Auswirkungen \cellcolor{blue!20!}& \bf Massnahmen \cellcolor{blue!20!}& \bf Status \cellcolor{blue!20!}\\ \hline
  1. & Die ausgew\"ahlten Technologien sind f\"ur die Programmierung der Applikation nicht geeignet, da Funktionen fehlen oder aufwendig zu implementieren sind.& Die Arbeit an den einzelnen Teilen des Programmes wird massiv aufwendiger. Versp\"atungen k\"onnen nicht ausgeschlossen werden. Evtl. m\"ussten sogar Teile der Applikation mit einer anderen Technologie neu geschriben werden.& Genaue Nachforschungen zu den Technologien anstellen. Prototypen und kleine Testapplikationen entwickeln.& erledigt\\ \hline
  2. & Die Applikation kann nicht rechtzeigit fertiggestellt werden, da zu viele Funktionen geplant sind, die Implementierung Probleme macht oder Mitglieder des Teams ausfallen.& Das Projekt kann nicht rechtzeitig fertiggestellt werden. Die Qualit\"at der Applikation leidet.& Planen der Ressourcen.& erledigt\\ \hline
  3. & Die Kompatibilit\"at mit allen Betriebssystemen kann nicht gew\"ahrleistet werden, da manche features auf einer Plattform viel schwieriger zu implemntieren sind als auf einer anderen.& Das Programm l\"auft nicht auf allen Systemen gleich stabil. M\"oglicherweise sind auch einige Funktionen nicht auf allen Plattformen verf\"ugbar. Zeitaufwendige Implementierung f\"ur verschidene Plattformen k\"onnten auch zu Versp\"atungen f\"uhren.& Verwendung einer Systemunabh\"angigen Sprache (Python). Nachvorschungen anstellen zu den Modulen, die f\"ur die Funktionalit\"at entscheidend sind. Im schlimmsten Falle k\"onnten wir auch den Support f\"ur eine Plattform fallen lassen.& erledigt\\ \hline
  4. & Die Applikation gen\"ugt den Anspr\"uchen in Sachen Stabilit\"at, Nutzerfreundlichkeit oder einem anderen Gebiet nicht.& Die Applikation funktioniert nicht wie gew\"unscht.& Die einzelnen Komponenten, ihre interoperabilit\"at und integration sowie das fertige Produkt sind zu testen& erledigt\\ \hline
\end{tabularx}
\section{QS-Plan}
\subsection{Vorgehen zur Qualitätssicherung}
Die Qualit\"at der Applikation haben wir durch Tests sichergestellt. Code-reviews haben keine stattgefunden.\\
Zum Testen der Applikation kamen vorallem Modultests zum Einsatz. Genauere Details zur Durchf\"uhrung der Tests k\"onnen \cite[4. Systemtests]{realisierung} entnommen werden.

Die Dokumente wurden jeweils nach dem Erstellen durchgelesen. Die Struktur der Vorlagedokumente gab jeweils vor, welche Informationen enthalten sein mussten.
\subsection{Qualitätsziele}
\newpage
\begin{tabularx}{\textwidth}{|l|l|X|X|}
  \hline
  \bf Nr. \cellcolor{blue!20!}& \bf Qualit\"atsmerkmal \cellcolor{blue!20!}& \bf Qualit\"atsziel \cellcolor{blue!20!}& \bf besondere QS-Massnahmen um das Ziel zu erreichen \cellcolor{blue!20!}\\ \hline
  \multicolumn{4}{|c|}{\bf aus Benutzersicht} \\ \hline
  1 & Funktionserf\"ullung & Die Applikation soll die im Konzeptbericht \cite{konzept} festgelegten Ziele erf\"ullen. & - \\ \hline
  2 & Effizienz & Die Applikation soll z\"ugig reagieren, die Benutzung soll sich durchgehen schnell anf\"hlen.& Evtl. m\"ussen einzelne Teil des Codee auf Geschwindigkeit optimiert werden.\\ \hline
  3 & Zuverl\"assigkeit& Die Applikation soll auch bei falschen Eingaben oder schnellen Eingabe-Abfolgen zuverl\"assig funktionieren und darf nicht abst\"urzen.& Die Tests sehen auch ungew\"ohnliche Szenarien vor, die die Applikation stark belasten und in dieser Form im Alltag nicht zu erwarten sind.\\ \hline
  4 & Benutzbarkeit& Die Oberfl\"ache des Programmes soll intuitiv sein. Wo das nicht direkt m\"oglich ist, soll sich die Applikation an gewohnte Bedienungs-Paradigmen halten.& - \\ \hline
  5 & Sicherheit& Die Applikation soll die Sicherheit und integrit\"at der Benutzerdaten nicht gef\"ahrden.& \\ \hline
  \multicolumn{4}{|c|}{\bf aus Entwicklersicht} \\ \hline
  6 & Erweiterbarkeit & Die Applikation wird objektorientiert entwickelt. Die Klassen werden systematisch und eindeutig bennt. Zudem sind die Funktionen fehlertollerant aufgebaut. Funktionen die von aussen genutzt werden k\"onnen enthalten eine Erl\"auterung.& - \\ \hline
  7 & Wartbarkeit & Die Applikation ist ausreichend dokumentiert. Ablaeufe sind direkt im Code dokumentiert, komplexe Vorg\"ange sind in der Dokumentation erkl\"art.& - \\ \hline
  8 & \"Ubertragbarkeit & - & - \\ \hline
  9 & Wiederverwendbarkeit & Die Komponenten der Applikation h\"ange nicht fest voneinander ab. Einzelne Komponenten (z.B. das Backend) k\"onnen auch in anderen Projekten wiederverwendet werden.& - \\ \hline
  \multicolumn{4}{|c|}{\bf Projektf\"uhrung} \\ \hline
  10 & Kommunikation unter den Beteiligten & Die Entwickler sprechen sich untereinander ab, damit nicht aneinander vorbeiprogrammiert ist. Treten Fragen zum Code eine anderen Mitgliedes des Teams auf, so k\"onnen diese gestellt werden.& - \\ \hline
  12 & Termineinhaltung & Die Termine aus dem Projektplan (siehe oben) werde eingehalten.& - \\ \hline
  13 & Projekdokumentation & Die Vorgegebenen Dokumente werden p\"unktlich eingereicht. Mails die zwischen den Entwicklern ausgetauscht werden werden aufgehoben.& - \\ \hline
\end{tabularx}
\subsection{Pr\"ufplan}
\begin{tabularx}{\textwidth}{|l|l|l|l|X|X|}
  \hline
  \bf Pr\"ufobjekt \cellcolor{blue!20!}& \bf Termin \cellcolor{blue!20!}& \bf Pr\"ufer \cellcolor{blue!20!}& \bf Pr\"ufmethode \cellcolor{blue!20!}& \bf Pr\"ufkriterien \cellcolor{blue!20!}& \bf Bemerkungen \cellcolor{blue!20!}\\ \hline
  Projektantrag		& & Lehrperson & Review & Alle Punkte behandelt (nach Vorlage), Rechtschreibung & \\ \hline
  Projektplan		& & Lehrperson & Review & Alle Punkte behandelt (nach Vorlage), Rechtschreibung, Tabelle ausgef\"ullt& \\ \hline
  Voranalysebericht	& & Lehrperson & Review & Alle Punkte behandelt (nach Vorlage), Rechtschreibung, Diagramme vorhanden, Auswertungen vorhanden & \\ \hline
  Konzeptbericht	& & Lehrperson & Review & Alle Punkte behandelt (nach Vorlage), Rechtschreibung, Diagramme vorhanden, Anwendungsf\"alle beschrieben, Klassendiagram vorhanden, DB-Schema vorhanden & \\ \hline
  Programm		& & Lehrperson & Tests & siehe \cite[4. Systemtests]{realisierung} & \\ \hline
  Realisierungsbericht	& & Lehrperson & Review & Alle Punkte behandelt (nach Vorlage), Rechtschreibung, Klassendiagram aktuell, DB-Schema aktuell, Benutzerhandbuch vorhanden, Supporthandbuch vorhanden, Tests erl\"autert, Code im Dokument & \\ \hline
  Einf\"uhrungsbericht	& & Lehrperson & Review & Alle Punkte behandelt (nach Vorlage), Rechtschreibung & \\ \hline
  Projektf\"uhrung	& & Lehrperson & Review & Alle Punkte behandelt (nach Vorlage), Rechtschreibung & \\ \hline
\end{tabularx}
\subsection{Pf\"ufmethoden}
\subsubsection{Review}
Die Dokumente wurden jeweils durchgelesen von mindestens einem Teammitglied. Zudem wurde beim Erstellen eine Rechtschreibekorrektur eingesetzt.
\subsubsection{Applikations-Tests}
siehe \cite[4. Systemtests]{realisierung} 
\subsection{Pr\"ufspezifikationen}
\subsubsection{Checklisten für die Prüfung der Projektdokumente}
\paragraph{Projektantrag}
\begin{tabularx}{\textwidth}{|l|X|X|}
  \hline
  {{\bf Nr.}}\cellcolor{blue!20!} & {\bf Pr\"ufkriterium}\cellcolor{blue!20!} & {\bf Pr\"ufergebnis, Bemerkungen}\cellcolor{blue!20!} \\ \hline
  1 & Rechtschreibung & ist nicht erfolgt \\ \hline
  2 & Alle Punkte behandelt & erf\"ullt \\ \hline
\end{tabularx}
\paragraph{Projektplan}
\begin{tabularx}{\textwidth}{|l|X|X|}
  \hline
  {{\bf Nr.}}\cellcolor{blue!20!} & {\bf Pr\"ufkriterium}\cellcolor{blue!20!} & {\bf Pr\"ufergebnis, Bemerkungen}\cellcolor{blue!20!} \\ \hline
  1 & Rechtschreibung & ist nicht erfolgt \\ \hline
  2 & Alle Punkte behandelt & erf\"ullt \\ \hline
  3 & Tabelle ausgef\"ullt & erf\"ullt \\ \hline
\end{tabularx}
\paragraph{Voranalysebericht}
\begin{tabularx}{\textwidth}{|l|X|X|}
  \hline
  {{\bf Nr.}}\cellcolor{blue!20!} & {\bf Pr\"ufkriterium}\cellcolor{blue!20!} & {\bf Pr\"ufergebnis, Bemerkungen}\cellcolor{blue!20!} \\ \hline
  1 & Rechtschreibung & erf\"ullt \\ \hline
  2 & Alle Punkte behandelt & erf\"ullt \\ \hline
  3 & Diagramme vorhanden & erf\"ullt \\ \hline
  4 & Auswertungen vorhanden & erf\"ullt \\ \hline
\end{tabularx}
\paragraph{Konzeptbericht}
\begin{tabularx}{\textwidth}{|l|X|X|}
  \hline
  {{\bf Nr.}}\cellcolor{blue!20!} & {\bf Pr\"ufkriterium}\cellcolor{blue!20!} & {\bf Pr\"ufergebnis, Bemerkungen}\cellcolor{blue!20!} \\ \hline
  1 & Rechtschreibung & erf\"ullt \\ \hline
  2 & Alle Punkte behandelt & erf\"ullt \\ \hline
  3 & Diagramme vorhanden & erf\"ullt \\ \hline
  4 & Anwendungsf\"alle beschrieben & erf\"ullt \\ \hline
  5 & Klassendiagram vorhanden & erf\"ullt \\ \hline
  6 & DB-Schema vorhanden & erf\"ullt \\ \hline
\end{tabularx}
\paragraph{Programm}
siehe \cite[4. Systemtests]{realisierung} 
\paragraph{Realisierungsbericht}
\begin{tabularx}{\textwidth}{|l|X|X|}
  \hline
  {{\bf Nr.}}\cellcolor{blue!20!} & {\bf Pr\"ufkriterium}\cellcolor{blue!20!} & {\bf Pr\"ufergebnis, Bemerkungen}\cellcolor{blue!20!} \\ \hline
  1 & Rechtschreibung & hat nicht stattgefunden\\ \hline
  2 & Alle Punkte behandelt & erf\"ullt\\ \hline
  3 & Klassendiagram aktuell& erf\"ullt \\ \hline
  4 & DB-Schema aktuell& erf\"ullt \\ \hline
  5 & Benutzerhandbuch vorhanden& erf\"ullt \\ \hline
  6 & Supporthandbuch vorhanden& erf\"ullt \\ \hline
  7 & Tests erl\"autert& erf\"ullt \\ \hline
  8 & Code im Dokument& erf\"ullt \\ \hline
\end{tabularx}
\paragraph{Einf\"uhrungsbericht}
\begin{tabularx}{\textwidth}{|l|X|X|}
  \hline
  {{\bf Nr.}}\cellcolor{blue!20!} & {\bf Pr\"ufkriterium}\cellcolor{blue!20!} & {\bf Pr\"ufergebnis, Bemerkungen}\cellcolor{blue!20!} \\ \hline
  1 & Rechtschreibung & erf\"ullt \\ \hline
  2 & Alle Punkte behandelt & erf\"ullt \\ \hline
\end{tabularx}
\paragraph{Projektf\"uhrung}
\begin{tabularx}{\textwidth}{|l|X|X|}
  \hline
  {{\bf Nr.}}\cellcolor{blue!20!} & {\bf Pr\"ufkriterium}\cellcolor{blue!20!} & {\bf Pr\"ufergebnis, Bemerkungen}\cellcolor{blue!20!} \\ \hline
  1 & Rechtschreibung & erf\"ullt \\ \hline
  2 & Alle Punkte behandelt & erf\"ullt\\ \hline
\end{tabularx}
\subsubsection{Testfalltabellen}
Testf\"alle wurden nur f\"ur den Code erstellt, nicht aber f\"ur die Dokumente. Die vertige Tabelle der Programmtests zusammen mit den Ergebnissen kann im Realisierungsbericht \cite{realisierung}eingesehen werden.
\begin{supertabular}{|l|l|p{3.5cm}|p{3.5cm}|p{3.5cm}|p{3.5cm}|}
\hline
\bf Nr. & \bf Afo-Nr.& \bf Anwendungsfall & \bf Ausgangs- situation & \bf Eingabedaten & \bf erwartetes Ergebnis \\ \hline
0 & 1& Normale Benutzung& Programm läuft nicht& Programm starten& Programm läuft \\ \hline
1 & 4 & Anwender f\"ugt einer Datei ein Tag hinzu. & Die Datei ~/.bashrc hat noch kein Tag. & Der Nutzer wechselt in das Verzeichnis ~, w\"ahlt die Datei .bashrc an und f\"ugt ihr im rechten Panel das Tag configfile hinzu. & Nach einem Neustart des Programmes und dem erneuten Selektieren der Datei wird das Tag in der Tags-Liste angezeigt. \\ \hline
2 & 2 & Normale Benutzung& Programm wurde gestartet& keine& Fenster mit Menu, Buttons und Eingabefeldern erscheint. \\ \hline
3 & - & Erstellen von Tags mit Sonderzeichen. & Das Tag /' existiert noch nicht. & Eine beliebige Datei wird selektiert, der Name des Tags wird in der Tags-Liste eingeegeben und durch 'speichern' festgehalten. Danach wird das Tag einer weiteren Datei hinzugef\"ugt. & Nach dem ersten Speichern, erscheint das Tag in der Liste aller Tags unten auf der rechten Seite des Programmes. Wird es einer zweiten Datei hnzugef\"ugt, so wird es NICHT dupliziert. \\ \hline
4 & 3& Benutzung auf nicht grafischen System& Programm wurde gestartet& keine& Command Prompt erscheint \\ \hline
5 & 5& Normale Benutzung& Programm wurde erfolgreich gestartet& User wählt eine Datei an& Tags werden angezeigt \\ \hline
6 & - & Ein Verzeichnis das eine Datei mit Sonderzeichen im Namen enth\"alt wird ge\"offnet und der Datei ein Tag hinzugef\"ugt. & Die Datei ~/''.txt hat keine Tags zugeordnet. & Die Datei ~/''.txt wird angesteuert und ihr ein beliebiges Tag hinzugef\"ugt. & Der Browser ist in der Lage das Verzeichnis mit der Datei zu \"offnen. Das Tag wird der Datei erfolgrech zugewiesen und bleibt erhalten. \\ \hline
7 & 6& ein Projekt soll Versioniert werden& Projektdaten liegen auf dem Filesystem& Der User aktiviert die Versionierung eines Tags& Die Dateien werden bei Veränderung und/oder nach einer Zeitlichen verzögerung kopiert. \\ \hline
8 & - & Wechseln der Ansicht. & Nach dem Programmstart wir die hierarchische Ansicht dargestellt. & Der Nutzer wechselt durch einen Klick auf den Button Tag (links unten im Programm), in die Tagansicht. Danach wechselt er \"uber das Ansichts-Men\"u wieder zur\"uck zur hierarchischen Ansicht. & Nach dem Klick auf den 'Tag'-Button wechselt das Programm zur Tag-Ansicht. Bei der \"uber das Men\"u ausgel\"osten Aktion wieder zur\"uck zur hierarchischen Ansicht. \\ \hline
9 & 8& Nach veränderung einer Datei soll von dieser ein Backup angelegt werden& Die Datei liegt auf dem Filesystem und der FileSystemListener wurde gestartet.& Der User verändert die Datei& Der FileSystemListener sendet ein \"Anderungsevent \\ \hline
10 & 9& Der User löscht eine Datei& Die Datei wurde in der Datenbank erfasst& Der User löscht eine Datei& Der FileSystemListener erkennt die Löschaktion und sendet ein Event an die Datenbank \\ \hline
11 & 10& Der User legt eine Datei an& Der FileSystemListener überwacht das Verzeichnis& Der User legt eine Datei an& Der FileSystemListener erkennt die neue Datei und sendet ein Event an das GUI damit dieses die Datei anzeigt \\ \hline
12 & -& Der User m\"ochte ein Backup anlegen& Dateien liegen auf dem Filesystem& Der User w\"ahlt die option Sichern an& \\ \hline
\end{supertabular}
\section{Konfigurationsmanagementplan (KM-Plan)}
\subsection{Aufzubewahrende Projektergebnisse und ggf. sonstige Dokumente}
\subsection{Ablagestruktur}
\subsection{Namenskonventionen}
\section{Konfigurationsidentifikation}
\end{document}
